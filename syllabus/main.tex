\documentclass[10pt]{article}

% |===| Package imports.

\usepackage{
  etex, graphicx, amssymb, amsmath, amstext, amsfonts, mathtools,
  multicol, pgfplots, array, listings, colortbl, ulem, ifthen, hyperref,
  url, xspace, multirow
}
\usepackage[T1]{fontenc}
\usepackage[utf8]{inputenc}
\usepackage{tikz}
\usepackage{fullpage}

\usetikzlibrary{shapes,arrows}

\renewcommand{\baselinestretch}{1.2}


\newcommand{\ita}[1]{\textit{#1}}
\newcommand{\tba}{\ita{TBA}\xspace}

% |===| Title related things.

\newcommand{\articleTitle}{
  CS:3820 Programming Language Concepts\\
  The University of Iowa, Spring 2016\\
  Course Syllabus
}
\newcommand{\pdfTitle}{CS}

% |===| Nice links.

\hypersetup{
  colorlinks=true,
  linkcolor=blue,
  urlcolor=blue,
  pdftitle={\pdfTitle}
}

% |===| Title page info.

\title{
  \articleTitle
}

\author{
  \begin{minipage}{.5\textwidth}
    \centering
    \textbf{Instructor:}\\
    Dr. Adrien Champion\\
    \href{mailto:adrien-champion@uiowa.edu}{adrien-champion@uiowa.edu}\\[2pt]
  \end{minipage}
  \begin{minipage}{.5\textwidth}
    \centering
    \textbf{Teaching assistant:}\\
    Richard Blair\\
    \href{mailto:richard-blair@uiowa.edu}{richard-blair@uiowa.edu}
  \end{minipage}
}


\date{}

\begin{document}


\maketitle

\section*{Lectures}
Every Tuesday and Thursday 2pm to 3:15pm, 106 GILH.


\section*{Office hours}
\begin{tabular}{l | l l}
  Adrien Champion & Tue 3:30pm-5:30pm and Thu 3:20pm-5pm & 101A MLH \\
  Richard Blair & Mon 12pm-1:30pm and Thu 5:00pm-6:30pm & \tba \\
\end{tabular}


\section*{Prerequisites}
\begin{tabular}{l l}
  CS$2230$ & Computer Science II: Data Structures, and \\
  CS$2630$ & Computer Organization or \\
  CS$2820$ & Object-Oriented Software Development
\end{tabular}


\section*{Communication}

This term we will be using \textbf{Piazza} for class discussion. The system is
highly catered to getting you help fast and efficiently from classmates, the
TA, and myself. Rather than emailing questions to the teaching staff, we
encourage you to post your questions on Piazza.

\centerline{
  \url{https://piazza.com/uiowa/spring2016/cs3820/home}
}

\noindent
The \textbf{lectures} and this syllabus are available on bitbucket on the
repository

\centerline{
  \url{https://bitbucket.org/AdrienChampion/plc/overview}
}

\noindent
Announcements on Piazza will let you know when the lectures are pushed to the
bitbucket repository. Still, \textbf{we recommend} you \ita{follow} the
repository to be notified as soon as possible of corrections, updates \ldots


\section*{Course overview}

In this class we will look at the design of modern programming languages. We
discuss a notion of \ita{abstract machine} and implementation of a language.
We will also define and study the concepts found in most languages: memory
management, type checking, abstraction mechanisms \ldots

We will use the \textbf{Rust} programming language in class and for most
assignments / projects. Rust is a new, ambitious programming language based on
relatively recent concepts. The result is a challenging, powerful language with
features such as built-in parallelism, very strong static guarantees, powerful
constructs, impressive efficiency. Rust benefits from a large community
considering its young age. Rust was originally introduced by Mozilla and
follows a completely open development process: many users end up contributing
directly to the standard / official libraries.\newline

No prior familiarity with any of these topics or Rust is assumed in this class.


\section*{Outline}

\begin{tabular}{c l | c l | c l}
  1 & Abstract Machines
    & 4 & Memory management
    & 7 & Control abstraction \\
  2 & Describing a language
    & 5 & Names and environment
    & 8 & Structuring data \\
  3 & Foundations
    & 6 & Control structures
    & 9 & Data abstraction \\
\end{tabular}


\section*{Textbooks and readings}

There is no required textbook although we recommend\\
\begin{tabular}{| c}
  \ita{Programming Languages: Principles and Paradigms},
  Maurizio Gabbrielli and Simone Martini
\end{tabular}\\
available in an electronic edition at the library.


\section*{Assignments, Exams and Grading}

You will be graded for the following work, which make up the respective percentage of your final grade:
\begin{itemize}\itemsep-1pt
  \item 4 Homework Assignments (Programming Assignments), 40\%
  \item In-class Midterm Exam, 20\%
  \item Final Project, 30\%
  \item Micro assignments, 10\%
\end{itemize}
Programming assignments will generally have a late deadline of two days. The
penalty for turning in a programming assignment between 0 and 24 hours after
the deadline 15\%, and 25\% for more than that. 48 hours after the deadline,
solutions will be posted and no late work accepted.\\
A micro assignment is a very small assignment, intended to take 20-30 minutes
at most. Think of it as a take-home quiz, intended to reinforce concepts from
class. Micro assignments will be posted at the end of some classes and are due
right at the start of the following class (within first 3 minutes). Micros will
not be accepted late.

The following cutoffs will be used to determine letter grades. In the ranges
below, x stands for your total score at the end of the semester. Final scores
near a cutoff will be individually considered for the next higher grade.
Plus(+) and minus(-) grades will also be given; their cutoffs will be
determined at the end of the semester.
\begin{center}
  \begin{tabular}{c | c}
    SCORE & GRADE \\\hline
    88 <= x < 100 & A \\
    75 <= x < 88 & B \\
    60 <= x < 75 & C \\
    50 <= x < 60 & D \\
    00 <= x < 50 & F \\
  \end{tabular}
\end{center}


\section*{Collaboration Policy}

The collaboration policy for programming assignments is that high-level
discussion of problems is OK, but detailed collaboration on solutions is
prohibited unless explicitly allowed. This means that unless explicitly
allowed, you should not look over another person's code from the class (this
includes writing the code together). You are not allowed to check that your
solution makes sense by reading another person's code. Solutions whose
similarity is too great to be a coincidence will be considered for possible
academic integrity violation. No collaboration at all is allowed on micro
assignments. I will protect your rights to a fair evaluation in this course
through enforcement: cases of suspected cheating will be reported to the Dean's
office, as required by CLAS policy.


\section*{Teaching Policies \& Resources}

\subsection*{Administrative Home}
The College of Liberal Arts and Sciences is the administrative home of this
course and governs matters such as the add/drop deadlines, the
second-grade-only option, and other related issues. Different colleges may have
different policies. Questions may be addressed to 120 Schaeffer Hall, or see
the CLAS Academic Policies Handbook at
\url{http://clas.uiowa.edu/students/handbook}.

\subsection*{Electronic Communication}
University policy specifies that students are responsible for all official
correspondences sent to their University of Iowa e-mail address (@uiowa.edu).
Faculty and students should use this account for correspondences (Operations
Manual, III.15.2, k.11).

\subsection*{Accommodations for Disabilities}
A student seeking academic accommodations should first register with Student
Disability Services and then meet with the course instructor privately in the
instructor's office to make particular arrangements. See
\url{http://sds.studentlife.uiowa.edu/} for more information.

\subsection*{Academic Honesty}
All CLAS students or students taking classes offered by CLAS have, in essence,
agreed to the College's Code of Academic Honesty: ``I pledge to do my own
academic work and to excel to the best of my abilities, upholding the IOWA
Challenge. I promise not to lie about my academic work, to cheat, or to steal
the words or ideas of others; nor will I help fellow students to violate the
Code of Academic Honesty.'' Any student committing academic misconduct is
reported to the College and placed on disciplinary probation or may be
suspended or expelled (CLAS Academic Policies Handbook).

\subsection*{CLAS Final Examination Policies}
The final examination schedule for each class is announced by the Registrar
generally by the fifth week of classes. Final exams are offered only during the
official final examination period. No exams of any kind are allowed during the
last week of classes. All students should plan on being at the UI through the
final examination period. Once the Registrar has announced the date, time, and
location of each final exam, the complete schedule will be published on the
Registrar's web site and will be shared with instructors and students. It is
the student's responsibility to know the date, time, and place of a final exam.

\subsection*{Making a Suggestion or a Complaint}
Students with a suggestion or complaint should first visit with the instructor
(and the course supervisor), and then with the departmental DEO. Complaints
must be made within six months of the incident (CLAS Academic Policies
Handbook).

\subsection*{Understanding Sexual Harassment}
Sexual harassment subverts the mission of the University and threatens the
well-being of students, faculty, and staff. All members of the UI community
have a responsibility to uphold this mission and to contribute to a safe
environment that enhances learning. Incidents of sexual harassment should be
reported immediately. See the UI Office of the Sexual Misconduct Response
Coordinator for assistance, definitions, and the full University policy.

\subsection*{Reacting Safely to Severe Weather}
In severe weather, class members should seek appropriate shelter immediately,
leaving the classroom if necessary. The class will continue if possible when
the event is over. For more information on Hawk Alert and the siren warning
system, visit the Department of Public Safety website.

\end{document}
